\documentclass{article}

\usepackage[margin=0.95in]{geometry}


\title{Moravec's Paradox}
\author{Krishna Agrawal, 19111031}
\date{\today}

\begin{document}
\maketitle

\section{Introduction}
Moravec's paradox on contrary to traditional assumptions, is the observation by Researchers of Artificial Intelligence and Robotics that, "reasoning requires very little computation, but sensorimotor skills require enormous computational resources". This  principle was articulated by \textbf{\textit{Hans Moravec, Rodney Brooks, Marvin Minsky}} and others in the 1980s.  

\section{The biological basis of human skills}
One possible explanation of the paradox, offered by Moravec, is based on evolution.\\\\
\textbf{\textit{Moravec writes:}}\\
"Encoded in the large, highly evolved sensory and motor portions of the human brain is a billion years of experience about the nature of the world and how to survive in it. The deliberate process we call reasoning is, I believe, the thinnest veneer of human thought, effective only because it is supported by this much older and much more powerful, though usually unconscious, sensorimotor knowledge. The oldest human skills are largely unconscious and so appear to us to be effortless. We are all prodigious olympians in perceptual and motor areas, so good that we make the difficult look easy."

Therefore, we should expect skills that appear effortless to be difficult to reverse-engineer and roughly proportional to the amount of time that skill has been evolving in animals, but the skills that require effort may not necessarily be difficult to engineer at all.

\subsection*{Some Examples}

\textbf{\textit{Skills that have been evolving for millions of years:}} recognizing a face, judging people's motivations, recognizing a voice, paying attention to things that are interesting; anything to do with perception, attention, visualization, motor skills, social skills and so on.\\\\
\textbf{\textit{Skills that have appeared more recently:}} mathematics, engineering, games, logic and scientific reasoning. These are hard for us because they are not what our bodies and brains were primarily evolved to do. 

\section{Historical influence on artificial intelligence}

In the early days of artificial intelligence research, leading researchers became optimistic that they would be able to create thinking machines in just a few decades. They were confident after successfully writing programs that used logic, solved algebra and geometry problems and played games like checkers and chess.\\

Generally Logic and algebra are difficult for people and are considered a sign of intelligence. Therefore many prominent researchers \textbf{\textit{assumed that, having solved the "hard" problems, the "easy" problems of vision and commonsense reasoning would soon fall into place.}}  But they were wrong, and one reason is that these problems are not easy at all, but incredibly difficult. Also, the fact that they had solved problems like logic and algebra was irrelevant, because these problems are extremely easy for machines to solve.


\end{document}